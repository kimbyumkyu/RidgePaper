Two-particle angular correlations have been measured in high-multiplicity $\sqrt{s} =13$ TeV proton-proton collisions by the ALICE collaboration. From the correlations, the yields of particle pairs at short-($\Delta\eta$ \approx 0) and long-range($1.6<\Delta\eta<1.8$) in rapidity have been extracted on the near side ($\Delta\varphi$ \approx 0).
The correlations and yields are reported as a function of transverse momentum($p_{\rm T}$) in the range $1 < p_{\rm T} < 5$ GeV/$c$.
Furthermore, the event-scale dependence has for the first time been studied by requiring the presence of high-$p_{\rm T}$ leading tracks and/or jets for varying $p_{\rm T}$ thresholds. 
The results demonstrate that the long-range “ridge” yield, related to the collective behavior of the produced system, is present in events with high-$p_{\rm T}$ processes. The magnitudes of the short- and long-range yields are found to grow with the event scale. 
The results have been compared to calculations with EPOS LHC and PYTHIA 8, with and without string-shoving interactions. It is found that while both models describe the qualitative trends in the data, EPOS LHC shows a better quantitative agreement, in particular for the $p_{\rm T}$ and event scale dependences.

%The observed azimuthal modulations of long-range correlations in pseudorapidity in small systems like pp or p-Pb collisions show strikingly similar features to those seen in heavy ion collisions. Many theoretical approaches to interpreting this effect have been developed. However, it is still unclear whether these long-range correlations are due to final or initial state effects. To further investigate these effects, we studied long-range correlations as a function of transverse momentum in very high multiplicity pp collisions at $\sqrt{s} =13$ TeV, collected with the high multiplicity event trigger during 2016 and 2017 with ALICE. In this talk, we present the near-side per-trigger yield at large pseudorapidity separation (ridge yield) as a function of transverse momentum in pp collisions at $\sqrt{s} =13$ TeV. The results are compared to previous measurements from CMS experiments. In addition, we present the ridge yield in events where harder fragmentation processes are present, to explore possible physical origins of long-range correlations.

