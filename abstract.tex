We report on the studies of two-particle angular correlations measured in high-multiplicity proton-proton collisions at $\sqrt{s} =13$ TeV with ALICE. 
The yields of two-particle angular correlations at short-($\Delta\eta$ $\sim$ 0) and long-range ($1.6 < |\Delta\eta| < 1.8$) in rapidity are measured at the near side ($\Delta\varphi \sim 0$).
The correlations and yields are reported as a function of charged-particle transverse momentum ($p_{\mathrm T}$) in $1 < p_{\mathrm T} < 5$ GeV/$c$.
Furthermore, the event-scale dependence has for the first time been studied by requiring the presence of high-$p_{\rm T}$ leading tracks and/or jets for varying $p_{\rm T}$ thresholds. 
The results demonstrate that the long-range ``ridge`` yield, related to the collective behavior of the produced system, is present in events with high-$p_{\mathrm T}$ processes. The magnitudes of the short- and long-range yields are found to grow with the event scale. 
The results have been compared to calculations with EPOS LHC and PYTHIA 8, with and without string-shoving interactions. It is found that while both models describe the qualitative trends in the data, EPOS LHC shows a better quantitative agreement, in particular for the $p_{\rm T}$ and event scale dependences.
