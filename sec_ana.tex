% !TEX root = paper.tex

\section{Experimental setup}
\label{sec:experiment}

Delivery of protons with world-highest energy by LHC at CERN makes it possible to generate various phenomena from their collisions. Recent center-of-mass energy of colliding two protons is increased up to $\sqrt{\it{s}}$~=~\unit[13]{TeV} during LHC Run2 period. Among data from proton-proton collisions at \unit[13]{TeV}, This paper describes analysis results obtained by using 2016 to 2018 data sets. The full description of ALICE detector in the LHC Run 2 can be found in Refs. \cite{Aamodt:2008zz,Abelev:2014ffa}. The present analysis mainly uses V0~\cite{Abbas:2013taa}, ITS (Inner Tracking System)~\cite{aliceITS} and TPC (Time Projection Chamber)~\cite{aliceTPC} detectors.

The V0 detector consists of two rings, V0-A and V0-C, each made of 32 scintillator tiles, covering the full azimuthal angle within 2.8$<\eta<$5.1 and -3.7$<\eta<$-1.7, respectively. The V0 provides trigger and estimation of event multiplicity. A sample of events including higher numbers of produced particles is obtained with a high multiplicity trigger, which is achieved by requiring higher amplitude of V0 Detector. 

The responsibility of reconstruction of charged track is up to the ITS and the TPC. The ITS is composed of three subsystems, Silicon Pixel Detector(SPD), Silicon Drift Detector (SDD) and Silicon Strip Detector (SSD). The ITS has an acceptance up to $|\eta|<$1.95 for single charged track reconstruction. The TPC, which is working inside solenoidal magnetic field of 0.5 T, has an acceptance up to $|\eta|<$0.9 for charged tracks reaching the outer radius of the TPC. The tracking of charged-particles is done with the combination of the ITS and the TPC, which enable the reconstruction of tracks down to 0.2 GeV/\it{c}\rm{} with $\sim$75\% efficiency.

The multiplicity class used in the present analysis is top 0-0.1\%, which denotes the most particle-abundant events including  $\sim$31 charged particles in the mid-rapidity region($|\eta|<$0.5). This analysis uses charged particles, whose reconstructed transverse momenta are larger than 0.2 GeV/\it{c}\rm{} in a fiducial region as $|\eta|<$0.9. 

\section{Analysis Procedure}
\label{sec:ana}

The two-particle correlation between trigger particle and associated particle is measured as function of relative pseudo rapidity($\Delta\eta$) and azimuthal angle($\Delta\varphi$). The following equation expresses the angular correlation function as associated yield per trigger particle as function of $\Delta\eta$ and $\Delta\varphi$ with a given transverse momentum($\it{p}_{\rm{T, trig}}$, $\it{p}_{\rm{T, assoc}}$) of trigger particles and associated particles.
%with the notation of $\it{p}_{\rm{T, trig}} > \it{p}_{\rm{T, assoc}}$.
\begin{eqnarray}
\frac{1}{N_{\rm{trig}}} \frac{ \rm{d}\it{}^{2} N_{\rm{pair}} }{ \rm{d} \Delta\eta \rm{d}\Delta\varphi} = B(0, 0)\frac{S(\Delta\eta, \Delta\varphi)}{B(\Delta\eta, \Delta\varphi)},
\end{eqnarray}
where the $N_{\rm{trig}}$ is the number of trigger particles in the corresponding event class. The signal distribution $S(\Delta\eta, \Delta\varphi)$ is constructed using two-particle correlation in the same event and the background distribution $B(\Delta\eta, \Delta\varphi)$ is constructed using two-particle correlation in mixed events having the same primary vertex and belonging to the same multiplicity class.

The quantitative study of ridge is done with $\Delta\varphi$ distribution at large $\Delta\eta$ to achieve direct comparison of ridges between different event classes and transverse momentum intervals. The large $\Delta\eta$ range is defined as 1.6$<|\Delta\eta|<$1.8, which allows non-flow effects, mainly coming from the jet, not to contribute to the $\Delta\varphi$ distribution.
\begin{eqnarray}
\frac{1}{N_{\rm{trig}}} \frac{ \rm{d}\it{}N_{\rm{pair}} }{ \rm{d}\Delta\varphi } = \int_{|\Delta \eta|>1.6} \rm{d} \Delta \eta \frac{1}{\it{N}_{\rm{trig}}} \frac{ \rm{d}\it{}^{2} N_{\rm{pair}} }{ \rm{d}\Delta\eta \rm{d}\Delta\varphi}
\end{eqnarray}
The baseline of the correlations is subtracted by implementing Zero-Yield-At-Minimum (ZYAM) procedure. The minimum yield ($C_{\rm{ZYAM}}$) at the minimum $\Delta\varphi$($\Delta\varphi_{\rm{min}}$) of the $\Delta\varphi$ distribution are obtained from the function, which fits the $\Delta\varphi$ distribution with Fourier series up to the third harmonic. Subtracting $C_{\rm{ZYAM}}$ from the $\Delta\varphi$ distribution makes the yield at $\Delta\varphi_{\rm{min}}$ zero in order to describe the shape to find the minimum to be used for integral of the yield. The associated yield of the ridge($Y^{\rm{assoc}}$) is obtained by integrating the near-side peak of the $\Delta\varphi$ distribution over $|\Delta\varphi|<|\Delta\varphi_{\rm{min}}|$ after ZYAM.
\begin{eqnarray}
Y^{\rm{assoc}} = \int_{|\Delta \varphi| < |\Delta\varphi_{\rm{min}}| } \rm{d} \Delta\varphi \frac{1}{\it{N}_{\rm{trig}}} \frac{ \rm{d}\it{}N_{\rm{pair}} }{ \rm{d}\Delta\varphi } 
\end{eqnarray}

The ridge yield is further studied with various event selections regarding hard processes. The event selection is applied by requiring minimum transverse momentum of leading track or jet reconstructed in the mid-rapidity. The leading track is accepted within $|\Delta\eta|<0.9$ and reconstructed jet, which are made with anti-$k_{\rm{T}}$ algorithm with cone radius as 0.4, is accepted within $|\Delta\eta|<0.4$. The high $\it{p}_{\rm{T}}$ track and jet mostly generated from the hard scatterings in the initial collisions of hadrons so that tagging of events with transverse momentum of leading track or jet allows us to access to the high momentum transfer from the collisions to the initial partons.

Monte Carlo simulation with PYTHIA8 event generator and with particle transport through the detector using GEANT simulation has been used to estimate the charged single particle efficiency and the contamination from the non-primary particle. The corrections have been tested by comparing the distributions, which are constructed with generated true particles and reconstructed particles with detector responses. The several percentage of non-closure has been considered into systematic uncertainty.

\section{Systematic Uncertainty}
\label{sec:uncertainties}



The background distribution is constructed using several events having same primary vertex. The effect of construction of background distribution is estimated by varying the primary vertex interval from 2 cm to 1 cm. The estimated effect is 6-10\%.

The $\Delta\eta$ projection range for construction of long-range $\Delta\varphi$ is sensitive to contamination of non-flow effect. The effect of  $\Delta\eta$ projection range is estimated by varying the projection range. The estimated effect is 10-15\% for $\it{p}_{\rm{T}}>$1.0 GeV/\it{c}\rm{} and 20\% for 0.5$<\it{p}_{\rm{T}}<$1.0 GeV/\it{c}\rm{}. 

The M.C. closure test for efficiency correction results in $\sim$4\% discrepancy. The discrepancy is considered into systematic uncertainty.

The effect of primary vertex selection along the beam axis is estimated by varying the selection range of primary vertex from $|z_{vtx}|<$ 8 cm to $|z_{vtx}|<$ 6 cm. The estimated effect is $\sim$4\%. The effect of rejection of pile-up events. is estimated by varying the methodology of pile-up rejection. The estimated effect is $\sim$4\%. The effect of track selection is estimated by varying a few selection criteria included in track selection.The estimated effect is $\sim$5\%. 


\begin{table}[!h]
\centering
\caption{ Summary of the systematic uncertainties. See text for details.}
\begin{tabular}{ c|c }
%\begin{tabular}{c || c}
\hline
Source &  Uncertainty \\ \hline
Event mixing & 6-10\% \\  \hline
$\Delta\eta$ projection range & 10-15\% \\ \hline
M.C. closure & 4\% \\ \hline
Primary vertex & 4\% \\ \hline
Pileup Cut & 4\% \\ \hline
Track selection & 5\% \\ \hline
Total & 14-20\% \\
\hline
\end{tabular}
%\caption{ Definition of multiplicity class in this analysis. Reference definition of multiplicity coud be found in the following link; \url{https://twiki.cern.ch/twiki/bin/viewauth/ALICE/ReferenceMult}  }
\end{table}



