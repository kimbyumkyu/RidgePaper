% !TEX root = paper.tex

\section{Experimental setup}
\label{sec:experiment}

%LHC at CERN produces various interesting phenomena including the ridge effect that is the main topic of the document in pp collisions with the highest center-of-mass energy in the world.
%Recent center-of-mass energy in pp collisions reaches up to $\sqrt{s} = 13$~TeV during the last LHC Run 2 period. 
The analysis is based on the data sets in pp collisions at $\sqrt{s} = 13$~TeV collected from 2016 to 2018 in the LHC Run 2 period.  The full description of ALICE detector in the LHC Run 2 can be found in Refs.~\cite{Aamodt:2008zz,Abelev:2014ffa}. The present analysis mainly utilizes V0~\cite{Abbas:2013taa}, ITS (Inner Tracking System)~\cite{aliceITS} and TPC (Time Projection Chamber)~\cite{aliceTPC} detectors.


The V0 detector consists of two rings, V0-A and V0-C, each made of 32 scintillator tiles, covering the full azimuthal angle within 2.8$<\eta<$5.1 and -3.7$<\eta<$-1.7, respectively. The V0 provides trigger and estimation of event multiplicity. A sample of events including higher numbers of produced particles is obtained with a high multiplicity trigger, which is achieved by requiring some thresholds on signal amplitudes of energy deposit by charged particles in the V0 detector.

Charged particles are reconstructed by the ITS and TPC, which are working in a constant solenoidal magnetic field of 0.5~T.  The ITS is composed of three sub-systems, Silicon Pixel Detector(SPD), Silicon Drift Detector (SDD) and Silicon Strip Detector (SSD). The ITS and TPC covering the full azimuthal region have acceptances up to $|\eta| < 1.4$ and 0.9, respectively, for detection of charged particles with a primary vertex in $|z_\mathrm{vtx}|<10$~cm. The tracking of charged particles is done with the combined information of the ITS and TPC that enables the reconstruction of tracks down to 0.2~GeV/$c$ with a $\sim $75\% efficiency.


%The multiplicity class is classified by the signal amplitude of the V0 detector.
The multiplicity class used in the present analysis is the top 0.1\% signal amplitude events by the V0 detector for the set of minimum bias events at ALICE. Corresponding number of charged particles is reported as $\sim$31 in the mid-rapidity region ($|\eta|<0.5$) that is almost 5 times larger than the minimum-bias events.  Statistics of the high multiplicity event benefits from a dedicated high multiplicity trigger that was recently implemented at ALICE  in the LHC Run 2 period.  
 


\section{Analysis Procedure}
\label{sec:ana}

The two-particle correlation between trigger particle and associated particle is measured as function of relative pseudo rapidity($\Delta\eta$) and azimuthal angle($\Delta\varphi$). The following equation expresses the correlation function as associated yield per trigger particle as function of $\Delta\eta$ and $\Delta\varphi$ with a given transverse momentum($\it{p}_{\rm{T, trig}}$, $\it{p}_{\rm{T, assoc}}$) of trigger particles and associated particles.
%with the notation of $\it{p}_{\rm{T, trig}} > \it{p}_{\rm{T, assoc}}$.
\begin{eqnarray}
\frac{1}{N_{\rm{trig}}} \frac{ \rm{d}\it{}^{2} N_{\rm{pair}} }{ \rm{d} \Delta\eta \rm{d}\Delta\varphi} = B(0, 0)\frac{S(\Delta\eta, \Delta\varphi)}{B(\Delta\eta, \Delta\varphi)},
\end{eqnarray}
where the $N_{\rm{trig}}$ is the number of trigger particles in the corresponding multiplicity class. The signal distribution $S(\Delta\eta, \Delta\varphi)$ is constructed using two-particle correlation in the same event and the background distribution $B(\Delta\eta, \Delta\varphi)$ is constructed using two-particle correlation in mixed events having the same primary vertex and belonging to the same multiplicity class considering the change of acceptance effect with respect to the multiplicity class and primary vertex.

The quantitative study of ridge is done with $\Delta\varphi$ distribution at large $\Delta\eta$ to achieve direct comparison of ridges between different event classes and transverse momentum intervals. The large $\Delta\eta$ range is defined as 1.6$<|\Delta\eta|<$1.8, which allows non-flow effects, mainly coming from the jet, not to contribute to the $\Delta\varphi$ distribution.
\begin{eqnarray}
\frac{1}{N_{\rm{trig}}} \frac{ \rm{d}\it{}N_{\rm{pair}} }{ \rm{d}\Delta\varphi } = \int_{|\Delta \eta|>1.6} \rm{d} \Delta \eta \frac{1}{\it{N}_{\rm{trig}}} \frac{ \rm{d}\it{}^{2} N_{\rm{pair}} }{ \rm{d}\Delta\eta \rm{d}\Delta\varphi}
\end{eqnarray}
The baseline of the correlations is subtracted by implementing Zero-Yield-At-Minimum (ZYAM) procedure. The minimum yield ($C_{\rm{ZYAM}}$) at the minimum $\Delta\varphi$($\Delta\varphi_{\rm{min}}$) of the $\Delta\varphi$ distribution are obtained from the function, which fits the $\Delta\varphi$ distribution with Fourier series up to the third harmonic. Subtracting $C_{\rm{ZYAM}}$ from the $\Delta\varphi$ distribution makes the yield at $\Delta\varphi_{\rm{min}}$ zero in order to describe the shape to find the minimum to be used for integral of the yield. The associated yield of the ridge($Y^{\rm{assoc}}$) is obtained by integrating the near-side peak of the $\Delta\varphi$ distribution over $|\Delta\varphi|<|\Delta\varphi_{\rm{min}}|$ after ZYAM.
\begin{eqnarray}
Y^{\rm{assoc}} = \int_{|\Delta \varphi| < |\Delta\varphi_{\rm{min}}| } \rm{d} \Delta\varphi \frac{1}{\it{N}_{\rm{trig}}} \frac{ \rm{d}\it{}N_{\rm{pair}} }{ \rm{d}\Delta\varphi } 
\end{eqnarray}

The ridge yield is further studied with various event selections regarding hard processes. The event selection is applied by requiring minimum transverse momentum of leading track or jet reconstructed in the mid-rapidity. The leading track is accepted within $|\Delta\eta|<0.9$ and reconstructed jet, which are made with anti-$k_{\rm{T}}$ algorithm with cone radius as 0.4, is accepted within $|\Delta\eta|<0.4$. The transverse momentum of jet is also corrected with underlying event density($\rho \cdot A$). The high $\it{p}_{\rm{T}}$ track and jet mostly generated from the hard scatterings in the initial collisions of hadrons so that tagging of events with transverse momentum of leading track or jet allows us to access to the high momentum transfer from the collisions to the initial partons.

Monte Carlo simulation with PYTHIA8 event generator and with particle transport through the detector using GEANT simulation has been used to estimate the charged single particle efficiency and the contamination from the non-primary particle. 

%The corrections have been tested by comparing the distributions, which are constructed with generated true particles and reconstructed particles with detector responses. The several percentage of non-closure has been considered into systematic uncertainty.

\section{Systematic Uncertainty}
\label{sec:uncertainties}
The systematic uncertainty is evaluated by varying the event selection, track selection, yield extraction procedures and efficiency correction. Each source is separated into individual parts. The uncertainty from the event selection can be divided into Pile-up rejection and primary vertex selection. The uncertainty from the track selection is estimated by changing the selection criteria. The uncertainty from the yield extraction procedures can be divided into long-range definition and ZYAM procedure. The uncertainty from the efficiency correction is estimated by enlarging the event mixing bins, which is relevant to acceptance correction. Finally, The uncertainty from the application of track efficiency correction is estimated comparing the results constructed by efficiency corrected tracks with the results constructed by true particles. The uncertainty of $Y^{\rm{assoc}}(\it{p}_{\rm{T, trig}},\,\it{p}_{\rm{T, assoc}})$ is summarized by averaging the each uncertainty for a given $\it{p}_{\rm{T, trig}},\,\it{p}_{\rm{T, assoc}}$. The uncertainty of $Y^{\rm{assoc}}(\it{p}_{\rm{T, Lead}})$ or $Y^{\rm{assoc}}(\it{p}_{\rm{T, Jet}})$ is summarized by averaging the each uncertainty for a given $\it{p}_{\rm{T, Lead}}$ or $\it{p}_{\rm{T, Jet}}$ selection.
\begin{table}[!h]
\caption{ The relative systematic uncertainty(\%) of the associated yield spectrum as function of $(\it{p}_{\rm{T, trig}},\,\it{p}_{\rm{T, assoc}})$(second), $\it{p}_{\rm{T,Lead}}$(thrid) or $\it{p}_{\rm{T,jet}}$(fourth) selection in high multiplicity(0-0.1\%) }
\centering
\begin{tabular}{|c|<{\centering}m{8em}|<{\centering}m{6em}|<{\centering}m{6em}|}
\hline 
Sources & Uncertainty $Y^{\rm{assoc}}(\it{p}_{\rm{T, trig}},\,\it{p}_{\rm{T, assoc}})$ & Uncertainty $Y^{\rm{assoc}}(\it{p}_{\rm{T, Lead}})$ & Uncertainty $Y^{\rm{assoc}}(\it{p}_{\rm{T, Jet}})$ \\ \hline \hline
Pileup cut			& $\sim$0.8	&0-4		&0-4		\\ \hline
Primary Vertex		& $\sim$2.4	&1-12	&1-8		\\ \hline

Tracking			& $\sim$4.0 	&2		&2		\\ \hline

ZYAM			& $\sim$5.1	&2		&2		\\ \hline
$\Delta\eta$ range	& $\sim$4.5	&3-7		&3-9		\\ \hline

Mixing			& $\sim$4.4	&2-8		&1-16	\\ \hline

M.C. non-closure*	& $\sim$2.5 	&1		&1		\\ \hline
Total 			& $\sim$9.7	&4-16	&4-22	\\ 
\hline 
\end{tabular}

\end{table}

The uncertainty from the pileup rejection is estimated by inspecting the changes of results with different rejection methodologies from the default one. The estimated uncertainty is 1\% without $\it{p}_{\rm{T, Lead}}$ or $\it{p}_{\rm{T, Jet}}$ selection. The uncertainty is increased up to 4\% with $\it{p}_{\rm{T, Lead}}$ or $\it{p}_{\rm{T, Jet}}$ selection.

The uncertainty from the primary vertex selection along the beam axis is estimated by varying the selection range of primary vertex from $|z_{vtx}|<$ 8 cm to $|z_{vtx}|<$ 6 cm. The narrower primary vertex selection allows one to accept more forward tracks, which could be very sensitive to select long-range side. The estimated uncertainty is increased up to 12\% from 1\%, which is obtained without the event scale selection.

The uncertainty from the track selection is estimated by varying the track selection methodologies from the one optimized for refining the angular track distribution to the other optimized for particle identification. The estimated uncertainty for track selection is 4\% by averaging the uncertainty for whole transverse momentum bins and 2\% for 1$<\it{p}_{\rm{T}}<$2 GeV/\it{c}\rm{} tracks.

The uncertainty from the ZYAM procedure is estimated by varying the finding-scope for minimum position. The estimated uncertainty for track selection is 5.1\% by averaging the uncertainty for whole transverse momentum bins and 2\% for 1$<\it{p}_{\rm{T}}<$2 GeV/\it{c}\rm{} tracks, which is not largely affected by the event scale selection.

The uncertainty from the long $\Delta\eta$ range is estimated by varying the $\Delta\eta$ range. The uncertainty might include effects from the longitudinal de-correlations and possible non-flow contamination, which is thought be small. The estimated uncertainty is increased up to 9\% from 3\%, which is obtained by averaging the whole transverse momentum range.

The uncertainty from the mixing pool bin size of primary vertex is estimated by varying the bin size from 2.0 cm to 1.0 cm. The estimated uncertainty is 1-16\%

*The M.C. closure test for efficiency correction results in $\sim$2.5\% discrepancy. The discrepancy is considered into systematic uncertainty.



\iffalse


\begin{table}[!h]
\centering
\caption{ Summary of the systematic uncertainties. See text for details.}
\begin{tabular}{ c|c }
%\begin{tabular}{c || c}
\hline
Source &  Uncertainty \\ \hline
Event mixing & 6-10\% \\  \hline
$\Delta\eta$ projection range & 10-15\% \\ \hline
M.C. closure & 4\% \\ \hline
Primary vertex & 4\% \\ \hline
Pileup Cut & 4\% \\ \hline
Track selection & 5\% \\ \hline
Total & 14-20\% \\
\hline
\end{tabular}
%\caption{ Definition of multiplicity class in this analysis. Reference definition of multiplicity coud be found in the following link; \url{https://twiki.cern.ch/twiki/bin/viewauth/ALICE/ReferenceMult}  }
\end{table}
\fi




