

\section{Experimental setup}
\label{sec:experiment}

%LHC at CERN produces various interesting phenomena including the ridge effect that is the main topic of the document in pp collisions with the highest center-of-mass energy in the world.
%Recent center-of-mass energy in pp collisions reaches up to $\sqrt{s} = 13$~TeV during the last LHC Run 2 period. 
The analysis is based on the data sets in pp collisions at $\sqrt{s} = 13$~TeV collected from 2016 to 2018 in the LHC Run 2 period. The full description of ALICE detector in the LHC Run 2 can be found in Refs.~\cite{Aamodt:2008zz,Abelev:2014ffa}. The present analysis mainly utilizes V0~\cite{Abbas:2013taa}, ITS (Inner Tracking System)~\cite{aliceITS} and TPC (Time Projection Chamber)~\cite{aliceTPC} detectors.


The V0 detector consists of two rings, V0-A and V0-C, each made of 32 scintillator tiles, covering the full azimuthal angle within $2.8 < \eta < 5.1$ and $-3.7 < \eta < -1.7$, respectively. The V0 provides trigger and estimation of event multiplicity. A sample of events including higher numbers of produced particles is obtained with a high multiplicity trigger, which is achieved by requiring 5 times of mean signal amplitudes of energy deposit by charged particles in the V0 detector.

Charged particles are reconstructed by the ITS and TPC, which are working in a constant solenoidal magnetic field of 0.5~T. The ITS is composed of three sub-systems, Silicon Pixel Detector (SPD), Silicon Drift Detector (SDD) and Silicon Strip Detector (SSD). The ITS and TPC covering the full azimuthal region have acceptances up to $|\eta| < 1.4$ and 0.9, respectively, for detection of charged particles with a primary vertex in $|z_\mathrm{vtx}| < 8$~cm. The tracking of charged particles is done with the combined information of the ITS and TPC that enables the reconstruction of tracks down to 0.15~GeV/$c$ with about 65\% efficiency.
%https://arxiv.org/pdf/1910.14400.pdf

%The multiplicity class is classified by the signal amplitude of the V0 detector.
The high multiplicity events used in the present analysis have the top 0.1\% signal amplitude (the 0--0.1\% multiplicity class) by the V0 detector with respect to the set of minimum bias events at ALICE. Corresponding number of charged particles is reported as $\sim$31 in the mid-rapidity region ($|\eta|<0.5$) that is almost 5 times larger than the minimum-bias events~\cite{}. Statistics of the high multiplicity event benefits from a dedicated high multiplicity trigger that was recently implemented at ALICE  in the LHC Run 2 period.  
 


\section{Analysis Procedure}
\label{sec:ana}

The two-particle correlations function between the trigger and associated particles are measured as functions of relative pseudorapidity ($\Delta\eta$) and azimuthal angle differences($\Delta\varphi$)
\begin{eqnarray}
\frac{1}{N_{\rm{trig}}} \frac{ \rm{d}\it{}^{2} N_{\rm{pair}} }{ \rm{d} \Delta\eta \rm{d}\Delta\varphi} = B(0, 0)\frac{S(\Delta\eta, \Delta\varphi)}{B(\Delta\eta, \Delta\varphi)}\quad,
\label{eq:corrfunction}
\end{eqnarray}
where  $\pttrig$ and $\ptassoc$ are the transverse momenta of  the trigger and associated particles, respectively, $N_\mathrm{trig}$ is the number of trigger particles, $N_\mathrm{pair}$ is the number of pairs of the trigger and associated particles, and $S (\Delta\eta, \Delta\varphi)$ and $B (\Delta\eta, \Delta\varphi)$ are the signal and background distributions, respectively. $S (\Delta\eta, \Delta\varphi)$ is constructed using two-particle correlations in the same event and $B(\Delta\eta, \Delta\varphi)$ is in mixed events. The background distribution is normalized by its value at $\Delta\eta$ and $\Delta\varphi = 0$, represented as $B (0,0)$. In mixed events, the primary vertices and multiplicity classes are required to be in the same range to avoid the acceptance effect with respect to the multiplicity class and primary vertex.

The ridge yields at large $\Delta\eta$ are extracted in different multiplicity classes and transverse momentum intervals. The large $\Delta\eta$ range is selected as $1.6<|\Delta\eta|<1.8$ where non-flow effects (mainly coming from jets) are negligible. In this region, the $\Delta\varphi$ distribution, or the so called per-trigger yield is expressed as
\begin{eqnarray}
\frac{1}{N_{\rm{trig}}} \frac{ \rm{d}\it{}N_{\rm{pair}} }{ \rm{d}\Delta\varphi } = \int_{|\Delta \eta|>1.6} \rm{d} \Delta \eta \frac{1}{\it{N}_{\rm{trig}}} \frac{ \rm{d}\it{}^{2} N_{\rm{pair}} }{ \rm{d}\Delta\eta d\Delta\varphi} \quad.
\end{eqnarray}

The baseline of the correlations is subtracted by implementing Zero-Yield-At-Minimum (ZYAM) procedure~\cite{Ajitanand:2005jj}. The minimum yield $(C_{\rm{ZYAM}})$ at the $\Delta\varphi$ ($\Delta\varphi_{\rm{min}}$) in the $\Delta\varphi$ distribution is obtained from the function, which fits the $\Delta\varphi$ distribution with Fourier series up to the third harmonic. Subtracting $C_{\rm{ZYAM}}$ from the $\Delta\varphi$ distribution makes the yield at $\Delta\varphi_{\rm{min}}$ zero in order to describe the shape to find the minimum to be used for integral of the yield. The ridge yield ($Y^{\rm{assoc}}$) is obtained by integrating the near-side peak of the $\Delta\varphi$ distribution over $|\Delta\varphi|<|\Delta\varphi_{\rm{min}}|$ after the ZYAM procedure
\begin{eqnarray}
Y^{\rm{assoc}} = \int_{|\Delta \varphi| < |\Delta\varphi_{\rm{min}}| } \rm{d} \Delta\varphi \frac{1}{\it{N}_{\rm{trig}}} \frac{ \rm{d}\it{}N_{\rm{pair}} }{ \rm{d}\Delta\varphi } \quad.
\end{eqnarray}

The ridge yield is further studied by exploiting the largest momentum transfer between the initial partons in a given event, which results from the hard scattering. In this article, larger momentum transfer denotes harder event scale. The larger momentum transfer is also expected to be connected with the shorter impact parameters in pp collisions in average~\cite{Sjostrand:1986ep,Frankfurt:2010ea} so that the event scale can be indirectly connected with the impact parameters.
% The ridge yield is further studied by exploiting the momentum transfer between the interacting partons,so called, event scale.  On average, increasingly hard parton interactions result from pp collisions with decreasing impact parameters between the two proton ///original version
The event scales are set by requiring minimum transverse momentum of leading track or reconstructed jet in mid-rapidity since hard scattering produces leading tracks with a high $p_\mathrm{T}$ or jets. The leading track is selected within $|\eta|<0.9$ and the full azimuthal angle. Jets are reconstructed with charged particles only (charged jets) for the anti-$k_{\rm{T}}$ algorithm and the resolution parameter $R = 0.4$. Jets are selected in $|\eta_\mathrm{jet}|<0.4$ and the full azimuthal angle. The transverse momentum of jets $p_\mathrm{T,\,jet}$ is corrected for underlying event density that is measured for the $k_{\rm{T}}$ algorithm with $R = 0.2$. 

Monte Carlo simulation with $\pythiam$ event generator and particle transport inside ALICE using GEANT3~\cite{Brun:1994aa} is used to correct the acceptance and the efficiency of the ALICE detectors. 

%The corrections have been tested by comparing the distributions, which are constructed with generated true particles and reconstructed particles with detector responses. The several percentage of non-closure has been considered into systematic uncertainty.

\section{Systematic Uncertainty}
\label{sec:uncertainties}
The systematic uncertainty is evaluated by varying the event selection, track selection, yield extraction procedures and detector efficiency correction. Each source is assumed to be uncorrelated. The uncertainty from the event selection can be divided into pile-up rejection and primary vertex selection. The uncertainty from the track selection is estimated by changing the selection criteria. The uncertainty from the yield extraction procedures can be categorized into long-range definition and ZYAM procedure. The uncertainty from the efficiency correction is estimated by enlarging the event mixing bins, which is relevant to the acceptance correction.
%Finally, The uncertainty from the application of the detector efficiency correction is estimated comparing the results constructed by efficiency corrected tracks with the results constructed. 
Finally, the uncertainty from the efficiency correction for charged tracks is estimated by comparing the correlations function constructed by reconstructed tracks from the detector responses with the correlation function constructed by generated particles from the event generator.
The uncertainty of $Y^{\rm{assoc}}(\pttrig,\,\ptassoc)$ is summarized by averaging each uncertainty for given $\pttrig$ and $\ptassoc$. The uncertainty of $Y^{\rm{assoc}}(\ptlead)$ and $Y^{\rm{assoc}}(\ptjet)$ is summarized by averaging the each uncertainty for a given $\ptlead$ and $\ptjet$ selection, respectively.
\begin{table}[h!]
\centering
\begin{tabular}{|c|c|c|c|}
\hline 
\multirow{2}{*}{Sources}  & \multicolumn{3}{c|}{Systematic uncertainty (\%)} \\\cline{2-4}
         & $Y^{\rm{assoc}}(\it{p}_{\rm{T, trig}},\,\it{p}_{\rm{T, assoc}})$ & $Y^{\rm{assoc}}(\it{p}_{\rm{T, Lead}})$ & $Y^{\rm{assoc}}(\it{p}_{\rm{T, Jet}})$ \\ \hline \hline
Pileup cut			& $\sim$0.8	&0--4		&0--4		\\ \hline
Primary Vertex		& $\sim$2.4	&1--12	&1-8		\\ \hline

Tracking			& $\sim$4.0 	&2		&2		\\ \hline

ZYAM			& $\sim$5.1	&2		&2		\\ \hline
$\Delta\eta$ range	& $\sim$4.5	&3-7		&3--9		\\ \hline

Mixing			& $\sim$4.4	&2--8		&1--16	\\ \hline

M.C. non-closure*	& $\sim$2.5 	&1		&1		\\ \hline
Total 			& $\sim$9.7	&4--16	&4--22	\\ 
\hline 
\end{tabular}
\label{tab:syst}
\caption{The relative systematic uncertainty of the associated yield spectrum are estimated for $Y^{\rm{assoc}}(\pttrig,\,\ptassoc)$, $Y^{\rm{assoc}}(\ptlead)$, and $Y^{\rm{assoc}}(\ptjet)$, respectively.}
\end{table}

The uncertainty from the pileup rejection is estimated by measuring the changes of results with different rejection criteria from the default one. The pileup rejection is achieved by reconstructing the multiple vertex in the individual event. The multiplicity of the reconstructed vertices determines whether the event is contaminated by pileup or not. The variation is set by the minimum multiplicity of reconstructed vertices from 5 to 3. 
%Pileup cut (multbins) -> (default pileup from the physics selection)
%Pileup cut (multbins) -> PileupMV (?)
The estimated uncertainty is 1\% without $\ptlead$ or $\ptjet$ selection. The uncertainty is increased up to 4\% with $\ptlead$ or $\ptjet$ selection. The uncertainty from the primary vertex selection along the beam axis is estimated by varying the selection range of primary vertex from $|z_\mathrm{vtx}|<$ 8 to 6 cm. The narrower primary vertex selection allows one to test the acceptance effect on the measurement. The estimated uncertainty is increased up to 12\% from 1\%.
The uncertainty from the track selection is estimated by varying the track selection criteria from the one optimized for refining the angular track distribution to the other optimized for particle identification. The variation is achieved by changing the mandatory SPD requirement and the maximal threshold for chi-square of the track fit.
%https://github.com/alisw/AliRoot/blob/master/ANALYSIS/ESDfilter/macros/AddTaskESDFilter.C
The estimated uncertainty for track selection is 4\% by averaging the uncertainty for whole transverse momentum bins and 2\% for $1<\it{p}_{\rm{T}}<2$ GeV/$c$ particles.

The uncertainty from the ZYAM procedure is estimated by varying the finding-scope for minimum position. The estimated uncertainty for track selection is 5.1\% by averaging the uncertainty for the whole transverse momentum bins and 2\% for $1<\it{p}_{\rm{T}}<2$ GeV/$c$ tracks, which is not largely affected by the event scale selection. The uncertainty from the $\Delta\eta$ projection range is estimated by varying the $\Delta\eta$ range. The uncertainty might include effects from the longitudinal de-correlations and possible non-flow contamination, which is thought be small. The estimated uncertainty is increased up to 9\% from 3\%, which is obtained by averaging the whole transverse momentum range. The uncertainty from the mixing pool bin size of primary vertex is estimated by varying the bin size from 2 to 1 cm. The estimated uncertainty is increased up to 8-16\% from 1-2\%.

%*The M.C. closure test for efficiency correction results in $\sim$2.5\% discrepancy. The discrepancy is considered into systematic uncertainty.



\iffalse


\begin{table}[!h]
\centering
\caption{ Summary of the systematic uncertainties. See text for details.}
\begin{tabular}{ c|c }
%\begin{tabular}{c || c}
\hline
Source &  Uncertainty \\ \hline
Event mixing & 6-10\% \\  \hline
$\Delta\eta$ projection range & 10-15\% \\ \hline
M.C. closure & 4\% \\ \hline
Primary vertex & 4\% \\ \hline
Pileup Cut & 4\% \\ \hline
Track selection & 5\% \\ \hline
Total & 14-20\% \\
\hline
\end{tabular}
%\caption{ Definition of multiplicity class in this analysis. Reference definition of multiplicity coud be found in the following link; \url{https://twiki.cern.ch/twiki/bin/viewauth/ALICE/ReferenceMult}  }
\end{table}
\fi




