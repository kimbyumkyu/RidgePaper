% !TEX root = paper.tex

\section{Introduction}

Measurements of two-particle angular correlations are typically performed in terms of two dimensional $\Delta\eta-\Delta\phi$ correlation functions, where $\eta$ is the pseudorapidity and $\phi$ is the azimuthal angle. Of particular interest in studies of possible novel  collective effects is the long-range (e.g., $|\Delta\eta| > 2.0$) structure of two-particle correlation functions, in which the effects of known sources such as resonance decays and fragmentation of high-momentum partons are known to be small. In most Monte Carlo (MC) event generators for proton-proton (pp) collisions, the typical sources of such long-range correlations are momentum conservation and away-side ($\Delta\phi$ $\approx$ $\pi$ ) jet correlations. Measurements in high-energy nucleus nucleus collisions have shown a long-range structure in the two-particle angular correlations functions, which has been attributed to the presence of the hot and dense matter formed~\cite{Romatschke:2007mq,Adams:2005dq,Adcox:2004mh,Arsene:2004fa,Back:2004je}. 

Collective effects are one of the key probes to study evolution of the hot and dense matter created in ultra-relativistic heavy-ion collisions. Strong collectivity observed in the azimuthal correlations of particles emitted over a wide pseudorapidity range, in high-energy nucleus-nucleus (AA) collisions at the BNL RHIC~\cite{Alver:2006wh} and the CERN LHC~\cite{AAbelev:2014pua,ATLAS:2011ah}, has indicated the formation of a strongly interacting quark gluon plasma (QGP) matter, which exhibits hydrodynamic behavior~\cite{}. 
The enhancement in the associated yield of two-particle correlations at small relative azimuthal angle ($\Delta\varphi$) that extends over a long-range of relative pseudorapidity ($\Delta\eta$), often so-called $\it{ridge}$,  is one of the crucial observables to study the collectivity~\cite{Chatrchyan:2012wg, Adam:2016izf}.
In recent years similar long-range collective azimuthal correlations are also observed for small systems with high final-state particle multiplicity such as proton-proton (pp)~\cite{Khachatryan:2015lva, Khachatryan:2016txc, Acharya:2019vdf} proton-nucleus (pA)~\cite{}, and lighter nucleus-nucleus systems~\cite{phenixnature}, revealing strong indications for collective flow with hydrodynamic characteristics even in small systems, even though the volume and lifetime of the medium produced are expected to be small. 
There are many theoretical attempts\cite{Mantysaari:2017cni, Zhao:2017rgg, Welsh:2016siu, Greif:2017bnr} to interpret the ridge considering hydrodynamics, saturation or other mechanisms, but a quantitative description of the full set of experimental data has not yet been achieved.




