% !TEX root = paper.tex

\section{Introduction}
\label{sec:intro}

Measurements of two-particle angular correlations are typically performed in terms of two dimensional $\Delta\eta-\Delta\phi$ correlation functions, where $\eta$ is the pseudorapidity and $\phi$ is the azimuthal angle. Of particular interest in studies of possible novel  collective effects is the long-range structure of two-particle correlation functions, in which the effects of known sources such as resonance decays and fragmentation of high-momentum partons are known to be small. In most Monte Carlo (MC) event generators for proton-proton (pp) collisions, the typical sources of such long-range correlations are momentum conservation and away-side ($\Delta\phi$ $\approx$ $\pi$ ) jet correlations.
The enhancement in the associated yield of two-particle correlations at small relative azimuthal angle ($\Delta\varphi$) that extends over a long-range of relative pseudorapidity ($\Delta\eta$), is dubbed ``ridge'' due to its shape in $\Delta\eta-\Delta\phi$ plot.
The shape of these $\Delta\phi$ correlations can be studied via a Fourier decomposition~\cite{Poskanzer:1998yz,Voloshin:2008dg}. The second and third order terms are the dominant harmonic coefficients $v_n$. The $v_n$ coefficients can be related to the collision geometry and density fluctuations of the colliding nuclei~\cite{Alver:2010gr,Alver:2010dn,ALICE:2011ab} and to the transport properties of the QGP in hydrodynamic models~\cite{Bernhard:2016tnd,Bernhard2019}.

Strong collectivity observed in the azimuthal correlations of particles emitted over a wide pseudorapidity range, in high-energy nucleus-nucleus collisions at RHIC~\cite{Adams:2005dq,Adcox:2004mh,Arsene:2004fa,Back:2004je} and LHC~\cite{Abelev:2012di, Abelev:2014pua, ATLAS:2011ah}, has indicated the formation of a strongly interacting quark gluon plasma (QGP) matter, which exhibits hydrodynamic behavior~\cite{Romatschke:2007mq}. The recent efforts are nowadays focused on constrainig the transport properties of the QGP in hydrodynamic models~\cite{Bernhard:2016tnd,Bernhard2019} along with few advanced experimental works~\cite{ALICE:2016kpq,Acharya:2017gsw,Acharya:2017zfg,Acharya:2020taj}.
In recent years similar long-range collective azimuthal correlations are also observed for small systems with high final-state particle multiplicity such as proton-proton (pp)~\cite{Aad:2015gqa,Khachatryan:2015lva,Khachatryan:2016txc,Acharya:2019vdf} proton-nucleus (pA)~\cite{Abelev:2012ola}, and lighter nucleus-nucleus systems~\cite{phenixnature}, revealing strong indications for collective flow with hydrodynamic characteristics even in small systems, even though the volume and lifetime of the medium produced are expected to be small. 

% TODO more pA measurements
The ridge structures in high-multiplicity pp and pPb events have been attributed to mechanisms that involve initial-state effects, such as gluon saturation~\cite{Dusling:2012cg} and colour connections~\cite{Sarma:2019teo} forming along the longitudinal direction and final-state effects, such as parton-induced interactions~\cite{Arbuzov:2011yr}, and collective effects arising in a high-density system possibly formed in these collisions~\cite{Zhao:2017rgg}. 
As a natural choice, hybrid models which implement both effects are geneally used in hydrodynamic simulations~\cite{Greif:2017bnr,Mantysaari:2017cni}. 
% IS only
% FS only
% IS + FS ::
The importance of the proton shape and its fluctuations to model the small system was recongnized in \cite{Mantysaari:2017cni}.
The hydrodynamics itself might not be the only mechanism of the observed collectivity was acknowledged in ~\cite{Zhao:2020pty}. 
The influence and interplay of initial state and final state effects are recently studied carefully for the first time in ~\cite{Greif:2019ygb}, pointing out that the details of the initial state are crucially important for the quantitative description of observables in small systems~\cite{Schenke:2019pmk}. 
The attempts to describe the collective effects systematically from the small to large systems are being made both for experiments~\cite{Acharya:2019vdf} and theory side~\cite{Greif:2019ygb}.
However, a quantitative description of the full set of experimental data has not yet been achieved.
The summary of various explanations for the observed correlations in these small systems are summarized in ~\cite{Strickland:2018exs}.

Furthermore, if collectivity in small systems is due to final state interactions, it should be possible to measure its effect on jets. Proving the presence of jet quenching will be another crucial milestone to demonstrate the existence of the final-state effect in high multiplicity pp collisions. The most of observables for the jet quenching in pp and pPb collisions didn't show any evidences so far~\cite{}. The difficulties are attributed to an ambiguous reference since the hard probes themselves are also enhanced by requiring high multiplicity in the event~\cite{}.

%Small system studies: A theory overview 1807.07191 Strickland:2018exs
%find one more one this regard.
The ATLAS experiment has recently shown that the ridge remains in events tagged with a Z-boson~\cite{Aaboud:2019mcw}, possibly with an accompanying jet.
The impact paramenter dependence on di-jet or multi-jet production in pp collisons was studied in \cite{Frankfurt:2003td}. $v_2$ will be different.
The microscopic model for collectivity, based on interacting string implemented in the PYTHIA8 Monte Carlo event generator so called ``shoving model''~\cite{Bierlich:2017vhg} , can qualitatively reproduce the CMS near-side ridge yield~\cite{Khachatryan:2016txc} and ATLAS Z-tagged ridge. This challenges the hydrodymic picture and predicts modifications to jet fragmentation
properties~\cite{Bierlich:2019ixq}.

% Bierlich:2017vhg (Shoving) Bierlich:2019ixq  
% add vn extractio??
% flow vn extraction : awayside ??
% 
% AA geometry dominant for v2 and what about in pp, b vs v2 or ridge..
% Frankfurt:2003td
% Aaboud:2019mcw Z-tagged ridge ATLAS

To further investigate these effects, we studied long-range correlations as a function of transverse momentum in very high multiplicity pp collisions at $\sqrt{s} =13$ TeV, collected with the high multiplicity event trigger during 2016 and 2017 with ALICE. In this article, we present the near-side per-trigger yield at large pseudorapidity separation as a function of transverse momentum. The results are compared to previous measurements from CMS experiments. In addition, we present the ridge yield in events where harder fragmentation processes are present, to explore possible physical origins of long-range correlations.
The experimental setup and measurements are described in Sec.~\ref{sec:experiment}. In Sec.~\ref{sec:ana} we present the analysis methods. The sources of systematic uncertainties are explained in Sec.~\ref{sec:uncertainties}. The results of the measurements are presented in Sec.~\ref{sec:results}. In Sec.~\ref{sec:theory} we present comparisons to model calculations.
Finally, Sec.~\ref{sec:summary} summarizes our new results.

