% !TEX root = paper.tex

\section{Introduction}
\label{sec:intro}

Strong collectivity observed in the azimuthal correlations of particles emitted over a wide pseudorapidity range, in high-energy nucleus-nucleus collisions at RHIC~\cite{Adams:2005dq,Adcox:2004mh,Arsene:2004fa,Back:2004je} and LHC~\cite{Abelev:2012di, Abelev:2014pua, ATLAS:2011ah}, indicates the formation of a strongly interacting matter called the quark-gluon plasma (QGP), which exhibits hydrodynamic behavior (see the reviews~\cite{Romatschke:2007mq,Jeon:2015dfa,Romatschke:2017ejr}). The recent efforts are focused on constraining the transport properties of the QGP in hydrodynamic models~\cite{Niemi:2015qia,Bernhard:2016tnd,Bernhard2019} along with few advanced experimental efforts~\cite{ALICE:2016kpq,Acharya:2017gsw,Acharya:2017zfg,Acharya:2020taj}.
In recent years similar collective behaviors via long-range azimuthal correlations are also observed for small systems with high final-state particle multiplicity such as proton-proton (pp)~\cite{Aad:2015gqa,Khachatryan:2015lva,Khachatryan:2016txc,Acharya:2019vdf}, proton-nucleus (pA)~\cite{Abelev:2012ola,Aad:2014lta,Aaboud:2016yar,Khachatryan:2016ibd}, and lighter nucleus-nucleus systems~\cite{PHENIX:2018lia,Aidala:2017ajz}, % strong indications for collective flow with hydrodynamic characteristics even in small systems, though the volume and lifetime of the medium produced by such a system are expected to be small~\cite{Busza:2018rrf,Nagle:2018nvi}. 
The observed long-range azimuthal correlations suggest flow with hydrodynamic characteristics even in these small systems, though the volume and lifetime of the medium produced by such a system are expected to be small~\cite{Busza:2018rrf,Nagle:2018nvi}. 

Measurements of two-particle angular correlations are typically performed in terms of two dimensional $\Delta\eta$--$\Delta\varphi$ correlation functions, where $\Delta\eta$ is the relative pseudorapidity and $\Delta\varphi$ is the azimuthal angle separation between both particles. Long-range structure of two-particle correlation functions is of particular interest in studies of possible novel collective effects, where the effects of known sources such as resonance decays and fragmentation of high-momentum partons are known to be small. For most pp Monte Carlo (MC) event generators, the typical sources of such long-range correlations are momentum conservation. %and away-side ($\Delta\varphi$ $\approx$ $\pi$) jet correlations.
The enhancement in the associated yield of two-particle correlations at small relative azimuthal angle ($\Delta\varphi$) that extends over a long-range of relative pseudorapidity ($\Delta\eta$), is dubbed ``ridge'' due to its characteristic shape in the $\Delta\eta$--$\Delta\varphi$ plane.
The shape of these $\Delta\varphi$ correlations can be studied via a Fourier decomposition~\cite{Poskanzer:1998yz,Voloshin:2008dg}. The second and third order terms are the dominant harmonic coefficients. In heavy-ion collisions harmonic coefficients can be related to the collision geometry and density fluctuations of the colliding nuclei~\cite{Alver:2010gr,Alver:2010dn,ALICE:2011ab} and to transport properties of the QGP in relativistic viscous hydrodynamic models~\cite{Gale:2012rq,Niemi:2015qia,Shen:2014vra,Bernhard:2016tnd,Bernhard2019}.

% TODO more pA measurements
The ridge structures in high-multiplicity pp and p--Pb events have been attributed to initial-state effects or final-state effects. The initial-state effects are usually referred to gluon saturation~\cite{Dusling:2012cg,Bzdak:2013zma} and color reconnections~\cite{Ortiz:2013yxa,Sarma:2019teo}, which are formed along the longitudinal direction. The final-state effects may contribute via parton-induced interactions~\cite{Arbuzov:2011yr} or collective effects due to a hydrodynamic behavior of the produced particles arising in a high-density system possibly formed in these collisions~\cite{Weller:2017tsr,Zhao:2017rgg}. 
Hybrid models implementing both effects are generally used in hydrodynamic simulations~\cite{Greif:2017bnr,Mantysaari:2017cni}. 
The proton shape and its fluctuations are also important to model the small systems~\cite{Mantysaari:2017cni}.
Nevertheless, the hydrodynamics itself might not be the only mechanism of the observed collectivity~\cite{Zhao:2020pty}. 
The influence and interplay of the initial state and final state effects have recently been studied carefully for the first time in ~\cite{Greif:2019ygb}, pointing out that the details of the initial state are crucially important for quantitative description of measurements in small systems~\cite{Schenke:2019pmk}. 
Attempts to describe the collective effects systematically from small to large systems are being made experimentally~\cite{Acharya:2019vdf} and theoretically~\cite{Greif:2019ygb}.
However, a quantitative description of the full set of experimental data has not been achieved yet.
A summary of various explanations for the observed correlations in these small systems is given in ~\cite{Strickland:2018exs,Loizides:2016tew,Nagle:2018nvi}.

Furthermore, if collectivity in small systems is due to final-state interactions, it is expected to have an effect on jets. Proving the presence of jet quenching will be another crucial milestone to demonstrate the existence of the QGP in high-multiplicity pp collisions. Most of the observables for the jet quenching in pp and p--Pb collisions do not show any clear evidence so far~\cite{Khachatryan:2016odn,Adam:2016jfp,Adam:2016dau,Acharya:2017okq}. The difficulties are attributed to an ambiguous reference since the hard probes themselves are also enhanced by requiring high-multiplicity event~\cite{Adam:2016jfp,Acharya:2018egz}.
%Small system studies: A theory overview 1807.07191 Strickland:2018exs
%find one more one this regard.
The ATLAS experiment recently showed that the ridge remains in events tagged with a Z-boson~\cite{Aaboud:2019mcw}, possibly with an accompanying jet.
The impact parameter dependence on dijet or multi-jet production in pp collisions was studied in~\cite{Frankfurt:2003td,Frankfurt:2010ea}.
The microscopic model for collectivity, based on interacting string implemented in $\pythiae$ event generator as so-called the ``String Shoving model''~\cite{Bierlich:2017vhg}, can qualitatively reproduce the CMS near-side ridge yield~\cite{Khachatryan:2016txc}. This challenges the hydrodynamic picture and predicts modifications to jet fragmentation
properties~\cite{Bierlich:2019ixq}.

% check this paper https://arxiv.org/abs/1910.13978 ATLAS pt cut...
% Bierlich:2017vhg (Shoving) Bierlich:2019ixq  
% add vn extractio??
% flow vn extraction : awayside ??
% 
% AA geometry dominant for v2 and what about in pp, b vs v2 or ridge..
% Frankfurt:2003td
% Aaboud:2019mcw Z-tagged ridge ATLAS

To further investigate these effects, we studied long-range correlations as a function of transverse momentum in very high-multiplicity pp collisions at $\sqrt{s} =13$ TeV, collected with the high-multiplicity event trigger during the LHC Run 2 data taking period. In this article, we present the near-side per-trigger yield at large pseudorapidity separation as a function of transverse momentum. The results are compared to previous measurements by the CMS collaboration~\cite{Khachatryan:2015lva}. In addition, we report the ridge yield in events, where a hard scattering took place, to explore possible physical origins of the long-range correlations.
The experimental setup and analysis method are described in Sec.~\ref{sec:experiment} and \ref{sec:ana}, respectively. The sources of systematic uncertainties are presented in Sec.~\ref{sec:uncertainties}. The results of the measurements and comparisons to model calculations are given in Sec.~\ref{sec:results}. Finally, Sec.~\ref{sec:summary} summarizes the obtained results.

 %h has indicated the formation of a strongly interacting quark-gluon plasma (QGP) matteras indicated the formation of a strongly interacting quark-gluon plasma (QGP) matter