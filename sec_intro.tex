% !TEX root = paper.tex

\section{Introduction}
\label{sec:intro}

In high-energy nucleus-nucleus collisions at RHIC~\cite{Adams:2005dq,Adcox:2004mh,Arsene:2004fa,Back:2004je} and LHC~\cite{Abelev:2012di, Abelev:2014pua, ATLAS:2011ah}, significant correlations are observed between particles emitted over a wide pseudorapidity range. These collective effects are indicative of the formation of a strongly interacting matter called the quark-gluon plasma (QGP), which exhibits hydrodynamic behavior (see the reviews~\cite{Romatschke:2007mq,Jeon:2015dfa,Romatschke:2017ejr}). 
Recent theoretical~\cite{Niemi:2015qia,Bernhard:2016tnd,Bernhard2019} and experimental~\cite{ALICE:2016kpq,Acharya:2017gsw,Acharya:2017zfg,Acharya:2020taj} advancements have contributed significantly to understanding the transport properties of the QGP.
Similar long-range correlations, typically interpreted as signatures of collectivity in heavy-ion collisions, are also observed in high-multiplicity proton-proton (pp)~\cite{Aad:2015gqa,Khachatryan:2015lva,Khachatryan:2016txc,Acharya:2019vdf}, proton-nucleus (pA)~\cite{Abelev:2012ola,Aad:2014lta,Aaboud:2016yar,Khachatryan:2016ibd}, and lighter nucleus-nucleus systems~\cite{PHENIX:2018lia,Aidala:2017ajz}.
Azimuthal correlations of particles separated by a large gap in pseudorapidity may suggest hydrodynamic behavior even in these small systems, although the volume and lifetime of the medium produced by such a colliding system are expected to be small, and there are other mechanisms which can produce similar flow-like signals~\cite{Busza:2018rrf,Nagle:2018nvi}.
%The observed long-range azimuthal correlations suggest flow with hydrodynamic characteristics even in these small systems, though the volume and lifetime of the medium produced by such a system are expected to be small~\cite{Busza:2018rrf,Nagle:2018nvi}. 

Measurements of two-particle angular correlations give information on many physical effects, including collectivity, hadronization, fragmentation, and femtoscopic effects~\cite{Lisa:2005dd}, and are typically quantified as a function of $\Delta\eta$, the relative pseudorapidity, and $\Delta\varphi$, the separation in azimuthal angle, of pairs of particles. The long-range structure of two-particle correlation functions is well suited to analyse collective effects, since it is not created by resonance decays and fragmentation of high-momentum partons. A typical source of long-range correlations in Monte Carlo pp generators is momentum conservation. %and away-side ($\Delta\varphi$ $\approx$ $\pi$) jet correlations.
The enhancement in the associated yield of two-particle correlations at small $\Delta\varphi$ that extends over a large $\Delta\eta$, is dubbed ``ridge'' due to its characteristic shape in the $\Delta\eta$--$\Delta\varphi$ plane.
The shape of these $\Delta\varphi$ correlations can be studied via a Fourier decomposition~\cite{Poskanzer:1998yz,Voloshin:2008dg}. The second and third order terms are the dominant harmonic coefficients. In heavy-ion collisions, harmonic coefficients can be related to the collision geometry and density fluctuations of the colliding nuclei~\cite{Alver:2010gr,Alver:2010dn,ALICE:2011ab} and to transport properties of the QGP in relativistic viscous hydrodynamic models~\cite{Gale:2012rq,Niemi:2015qia,Shen:2014vra,Bernhard:2016tnd,Bernhard2019}.

% TODO more pA measurements
The ridge structures in high-multiplicity pp and p--Pb events have been attributed to initial-state effects or final-state effects. The initial-state effects are usually referred to gluon saturation~\cite{Dusling:2012cg,Bzdak:2013zma} and color reconnections~\cite{Ortiz:2013yxa,Sarma:2019teo}, which are formed along the longitudinal direction. The final-state effects may contribute via parton-induced interactions~\cite{Arbuzov:2011yr} or collective effects due to a hydrodynamic behavior of the produced particles arising in a high-density system possibly formed in these collisions~\cite{Weller:2017tsr,Zhao:2017rgg}. 
Hybrid models implementing both effects are generally used in hydrodynamic simulations~\cite{Greif:2017bnr,Mantysaari:2017cni}. 
The proton shape and its fluctuations are also important to model the small systems~\cite{Mantysaari:2017cni}.
%Nevertheless, the hydrodynamics itself might not be the only mechanism of the observed collectivity~\cite{Zhao:2020pty,Schenke:2019pmk}. 
The influence and interplay of the initial state and final state effects should be further studied carefully with the inclusion of  the details of the initial state for quantitative description of measurements in small systems~\cite{Schenke:2019pmk,Schenke:2020mbo}. 
Attempts to describe the collective effects systematically from small to large systems are being made experimentally~\cite{Acharya:2019vdf} and theoretically~\cite{Schenke:2020mbo}.
However, a quantitative description of the full set of experimental data has not been achieved yet.
A summary of various explanations for the observed correlations in small systems is given in ~\cite{Strickland:2018exs,Loizides:2016tew,Nagle:2018nvi}.

Besides the hybrid models mentioned above, alternative approaches have been developed to describe collectivity in small systems. String shoving implemented in PYTHIA 8 describes long-range correlations by generating color strings which interact and repel each other in transverse direction, which results in a microscopic transverse pressure~\cite{Bierlich:2017vhg}. The microscopic model for collectivity, based on interacting strings implemented in $\pythiae$ event generator the so-called ``String Shoving model''~\cite{Bierlich:2017vhg}, can qualitatively reproduce the CMS near-side ridge yield~\cite{Khachatryan:2016txc}. This challenges the hydrodynamic picture and predicts modifications to jet fragmentation properties~\cite{Bierlich:2019ixq}.
EPOS LHC describes collectivity in small systems with a parameterized hydrodynamic evolution of the high-energy density region, so called ``core'', formed by many color string fields~\cite{Pierog:2013ria}.

If collectivity in small systems is due to final-state interactions, it is expected 
that these final-state interactions will affect also produced jets. Proving the presence of jet quenching would be another crucial evidence of the existence of the QGP in high-multiplicity pp collisions. However, so far there has been no sign of jet quenching observed in high-multiplicity pp and p--Pb collisions~\cite{Khachatryan:2016odn,Adam:2016jfp,Adam:2016dau,Acharya:2017okq}. Jet fragmentation can be studied in two-particle correlation functions by looking at short-range correlations around $(\Delta\eta$,$\Delta\varphi)=(0,0)$~\cite{Adam:2016tsv}.  
%The choice of a reference system can thus be ambiguous especially for pp since the hard probes themselves are enhanced by requiring high-multiplicity event~\cite{Adam:2016jfp,Acharya:2018egz}, also resulting in enhancement of probability to select events with high-$\pt$ jets.
%Small system studies: A theory overview 1807.07191 Strickland:2018exs
%find one more one this regard.
%The ATLAS experiment recently showed that the ridge remains in events tagged with a Z-boson~\cite{Aaboud:2019mcw}, possibly with an accompanying jet.
%The impact parameter dependence on dijet or multi-jet production in pp collisions was studied in~\cite{Frankfurt:2003td,Frankfurt:2010ea}.


% check this paper https://arxiv.org/abs/1910.13978 ATLAS pt cut...
% Bierlich:2017vhg (Shoving) Bierlich:2019ixq  
% add vn extractio??
% flow vn extraction : awayside ??
% 
% AA geometry dominant for v2 and what about in pp, b vs v2 or ridge..
% Frankfurt:2003td
% Aaboud:2019mcw Z-tagged ridge ATLAS

To further investigate the interplay of jet production and collective effects in small systems, we simultaneously studied long- and short-range correlations in very high-multiplicity pp collisions at $\sqrt{s} =13$ TeV, collected with the high-multiplicity event trigger during the LHC Run 2 data taking period. In this article, we present the near-side per-trigger yield at large pseudorapidity separation as a function of transverse momentum. The results are compared to the previous measurements by the CMS collaboration~\cite{Khachatryan:2015lva}. %In addition, we report the ridge yield in events, where a hard scattering took place, to explore possible physical origins of the long-range correlations.
In addition, we report the ridge yield and near-side jet-like correlations with event-scale selection. The event-scale selection is done by requiring a minimum transverse momentum of leading track or reconstructed jet in mid-rapidity, which is expected to bias the impact parameter of pp collisions to be smaller on average ~\cite{Sjostrand:1986ep,Frankfurt:2010ea}.
The experimental setup and analysis method are described in Sec.~\ref{sec:experiment} and \ref{sec:ana}, respectively. The sources of systematic uncertainties are discussed in Sec.~\ref{sec:uncertainties}. The results of the measurements and comparisons to model calculations are presented in Sec.~\ref{sec:results}. Finally, Sec.~\ref{sec:summary} summarizes the obtained results.


 %h has indicated the formation of a strongly interacting quark-gluon plasma (QGP) matteras indicated the formation of a strongly interacting quark-gluon plasma (QGP) matter