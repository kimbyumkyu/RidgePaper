% !TEX root = paper.tex

\section{Conclusions}
\label{sec:summary}

Long- and short-range correlations for pairs of charged particles with 1~$ < \pt < $~4~GeV/$c$ are studied in pp collisions at $\sqrt{s} = 13$~TeV with a focus on high-multiplicity events. The ridge and near-side jet yields are extracted and their event scale dependence have been studied. The obtained long-range ridge yields are compatible to those observed by the CMS Collaboration~\cite{Khachatryan:2015lva}.
%In addition, the $\pt$ and the $\it{p}_{\rm{T,Lead}}$ or $\it{p}_{\rm{T,Jet}}$ selection dependence of the long-range azimuthal correlations are measured.
The $\pythiashoving$ model describes the observed yields qualitatively but the yields it predicts decrease more rapidly with increasing $\pttrigassoc$ than those measured. On the other hand, the $\epos$ model gives a better description for the $\pttrigassoc$ dependence while overestimating the ridge yield for $\pttrigassoc<$~2 GeV/$c$. Finally, no long-range ridge is formed in the $\pythiam$ model.
%Furthermore the ridge yields are studied in events, where jets with a harder fragmentation are preferred by requiring the minimum value of the transverse momentum of leading particles in each event $\ptlead$ or the one of reconstructed jets $\ptjet$. The ridge structure still persist with both selections. The ridge yields increase as $\ptlead$ and $\ptjet$ increase. The latter, however, is more significant than the former selection. The results are compared with $\pythiashoving$ calculations, showing the enhancement while underestimating the absolute ridge yield.

The ridge yields are further studied in high-multiplicity events biased with additional event-scale selections, which impose a minimum transverse momentum cutoff on a leading track or jet. The ridge structure still persists with both selection criteria. The ridge yield increases as $\ptlead$ and $\ptjet$ increase. $\pythiashoving$ and $\epos$ estimate qualitatively the trends for the event-scale selections. However, the former underestimates and the latter overestimates it. The model predictions are also compared with the yield of the near-side jet-like correlation measured in the biased events. The evolution of the near-side jet yield as a function of event-scale $\pt$ is better captured by $\epos$, while the $\pythiashoving$ calculation tends to overshoot the data. 
%which results in opposite trends between the ridge yield and near-side jet-like correlations.
The results might open a new way of studying the impact parameter dependence of small systems with jet tagged events in the future and will help to constrain the physical origins of long-range correlations.
