% !TEX root = paper.tex

\section{Conclusions}
\label{sec:summary}
Long-range azimuthal correlations are studied in high-multiplicity pp collisions at $\sqrt{s} = 13$~TeV. These results are obtained in the relative pseudorapidity range $1.6 < |\Delta\eta| < 1.8$ and transverse momentum interval $1 < \pt < 5$~GeV/$c$ for the trigger and associated particles. 
The results are compatible to the measurements by the CMS collaboration~\cite{Khachatryan:2015lva}.
%In addition, the $\pt$ and the $\it{p}_{\rm{T,Lead}}$ or $\it{p}_{\rm{T,Jet}}$ selection dependence of the long-range azimuthal correlations are measured.
$\pythiashoving$ describes the yield qualitatively but its yield is decreased more rapidly than the data as $\pttrigassoc$ increases. On the other hand, $\epos$ gives better description for $\pt$ dependence while overestimating the ridge yield in the low $\pt<$2 GeV/$c$. $\pythiam$ does not describe the ridge in the long-range, which gives closure of the results against non-flow contamination.
%Furthermore the ridge yields are studied in events, where jets with a harder fragmentation are preferred by requiring the minimum value of the transverse momentum of leading particles in each event $\ptlead$ or the one of reconstructed jets $\ptjet$. The ridge structure still persist with both selections. The ridge yields increase as $\ptlead$ and $\ptjet$ increase. The latter, however, is more significant than the former selection. The results are compared with $\pythiashoving$ calculations, showing the enhancement while underestimating the absolute ridge yield.
Furthermore the ridge yields are studied with the event-scale selections through the minimum transverse momentum of jet or leading particle requirement, which are more prone of requiring harder processes in pp collisions. 
The ridge structure still persists with both selections. The ridge yields increase as $\ptlead$ and $\ptjet$ increase. The latter, however, is more significant than the former selection. The results are compared with $\pythiashoving$, $\epos$ and $\pythiam$ calculations, where $\pythiam$ also provides closure against non-flow contamination by estimate ridge yield to be zero with event-scale selections. $\pythiashoving$ and $\epos$ estimate qualitative trends for event-scale selection. The former underestimates the ridge yield, unlike the latter overestimating the ridge yield.
The results might open new way of studying impact parameter dependence of the small systems with jet tagged events in the future and will help to explore possible physical origins of long-range correlations.