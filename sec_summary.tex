% !TEX root = paper.tex

\section{Conclusions}
\label{sec:summary}

Long- and short-range correlations for pairs of charged particles with 1$ < \pt < $4~GeV/$c$ have been studied in pp collisions at $\sqrt{s} = 13$~TeV with a focus on high-multiplicity events. The ridge and near-side jet yields have been extracted and their event scale dependence have been studied. The obtained long-range ridge yields are compatible with the measurements by the CMS collaboration~\cite{Khachatryan:2015lva}.
%In addition, the $\pt$ and the $\it{p}_{\rm{T,Lead}}$ or $\it{p}_{\rm{T,Jet}}$ selection dependence of the long-range azimuthal correlations are measured.
$\pythiashoving$ describes the yield qualitatively but the predicted yield decreases more rapidly than the data as $\pttrigassoc$ increases. On the other hand, $\epos$ gives better description for $\pt$ dependence while overestimating the ridge yield for $\pt<$2 GeV/$c$. Finally, no long-range ridge is formed in $\pythiam$.
%Furthermore the ridge yields are studied in events, where jets with a harder fragmentation are preferred by requiring the minimum value of the transverse momentum of leading particles in each event $\ptlead$ or the one of reconstructed jets $\ptjet$. The ridge structure still persist with both selections. The ridge yields increase as $\ptlead$ and $\ptjet$ increase. The latter, however, is more significant than the former selection. The results are compared with $\pythiashoving$ calculations, showing the enhancement while underestimating the absolute ridge yield.
The ridge yields were further studied in high-multiplicity events biased with additional event scale selections, which imposed a minimum transverse momentum cutoff on a leading track or jet. The ridge structure still persists with both selections. The ridge yields increase as $\ptlead$ and $\ptjet$ increase. The results are compared with $\pythiashoving$, $\epos$ and $\pythiam$ calculations. $\pythiashoving$ and $\epos$ estimate qualitatively the trends for event-scale selection. However, the former underestimates the ridge yield and the latter overestimates the ridge yield. The models are further tested whether they reproduce the yield of the near-side jet-like correlation measured in the biased events.

Evolution of the near-side jet yield as a function event-scale $\pt$ is better captured by $\epos$, while $\pythiashoving$ tends to overshoot the data. 
%which results in opposite trends between the ridge yield and near-side jet-like correlations.
The results might open new way of studying impact parameter dependence of the small systems with jet tagged events in the future and will help to explore possible physical origins of long-range correlations.
