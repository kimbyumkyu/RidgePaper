% !TEX root = paper.tex

\section{Conclusions}
\label{sec:summary}


The $p_{\mathrm T}$ and the $\it{p}_{\rm{T,Lead}}$ or $\it{p}_{\rm{T,Jet}}$ selection dependence of the long-range azimuthal correlations are measured in very high-multiplicity pp collisions at center-of-mass energy $\sqrt{s} = 13$~TeV at the Large Hadron Collider. The spectrum of the ridge yield is measured in the relative pseudorapidity range $1.6 < |\eta| < 1.8$ for the transverse momentum range $1.0 < p_{\mathrm T} < 5.0$~GeV/$c$ and 0--0.1\% multiplicity event class. The results are compatible to the measurements from CMS experiment. One has to add more later about the different models.
The PYTHIA8 string shoving quantitatively estimates the spectrum while PYTHIA8 default shows zero yield. Further understandings of the correlations in small system are attempted by studying the ridge yield in events, where jets with harder fragmentations is to be present by requiring the $\it{p}_{\rm{T,Lead}}$ or $\it{p}_{\rm{T,Jet}}$. The ridge yield spectrum with hard fragmentations($p_{\mathrm T, jet} > 10$~GeV/$c$.) has shown the enhancement as the $\it{p}_{\rm{T,Lead}}$ or $\it{p}_{\rm{T,Jet}}$ increases. The results are compared with PYTHIA8 string shoving calculation, showing the enhancement while underestimating the absolute ridge yield.