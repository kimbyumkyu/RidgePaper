% !TEX root = paper.tex

\section{Conclusions}
\label{sec:summary}
Long-range azimuthal correlations are studied in high-multiplicity pp collisions at $\sqrt{s} = 13$~TeV. These results are obtained in the relative pseudorapidity range $1.6 < |\Delta\eta| < 1.8$ and transverse momentum interval $1 < \pt < 5$~GeV/$c$ for the trigger and associated particles. 
The results are compatible to the measurements by the CMS collaboration~\cite{Khachatryan:2015lva}.
%In addition, the $\pt$ and the $\it{p}_{\rm{T,Lead}}$ or $\it{p}_{\rm{T,Jet}}$ selection dependence of the long-range azimuthal correlations are measured.
$\pythiashoving$ describes the yield qualitatively but its yield is decreased more rapidly than the data as $\pttrigassoc$ increases. On the other hand, $\pythiam$ does not describe the ridge in the long-range as expected. The results from $\pythiam$ also assure that the measured ridge yield is not contaminated by non-flow effects.
Furthermore the ridge yields are studied in events, where jets with a harder fragmentation are preferred by requiring the minimum value of the transverse momentum of leading particles in each event $\ptlead$ or the one of reconstructed jets $\ptjet$. The ridge structure still persist with both selections. The ridge yields increase as $\ptlead$ and $\ptjet$ increase. The latter, however, is more significant than the former selection. The results are compared with $\pythiashoving$ calculations, showing the enhancement while underestimating the absolute ridge yield.
The results might open new way of studying impact parameter dependence of the small systems with jet tagged events in the future and will help to explore possible physical origins of long-range correlations.