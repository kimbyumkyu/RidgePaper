% !TEX root = paper.tex

\section{Conclusions}
\label{sec:summary}

%Two-particle angular correlations in a large pseudorapidity difference of $1.6<|\Delta\eta|<1.8$ are measured in very high-multiplicity pp collisions at $\sqrt{\it{s}}=13$ TeV with ALICE. The measured associated yield is found to be consistent with the previous results from CMS experiment. The ridge yield, for the first time, is observed with respect to the hard process. Furthermore, it is found to be independent of the hardness of events. This observation is important for the study of the origins of the ridge in small collision systems.



Ridge yields via long-range azimuthal correlations are measured with respect to the hard process in pp collisions at center-of-mass energy $\sqrt{s} = 13$~TeV.
The hard process is selected respectively by the transverse momenta of the leading track $p_\mathrm{T,\,Lead}$ and leading jet $p_\mathrm{T,\,Jet}$ for each event. $\p_\mathrm{T,\,Lead}$ is measured in a pseudorapidity region of $|\eta|<0.9$ for the full azimuthal angle. Jets are reconstructed with charged particles using the anti-$k_\mathrm{T}$ algorithm for a resolution parameter $R=0.4$ in a pseudorapidity region of $|\eta_\mathrm{jet}|<0.4$ for the full azimuthal angle. 
The ridge yield is measured in the relative pseudorapidity range $1.6 < |\Delta\eta| < 1.8$ for the transverse momentum range $1 < p_\mathrm{T} < 5$ GeV/$c$. The high multiplicity pp collisions are selected in terms of the energy deposit in the V0 detector of ALICE sensitive to charged-particle multiplicity in the forward region. 

The ridge behaviour is clearly appeared in high-multiplicity pp collisions that are relevant to the top 0.1\% events for the energy deposit in the V0. The result is compared to the CMS result and it is found to be compatible. The PYTHIA 8 String Shoving estimates the ridge yield quantitatively while PYTHIA 8 does not describe the ridge effect at all. Further study of the ridge behaviors in small system is done by measuring a ridge yield in events, where jets accompanying with the hard scattering process present by requiring thresholds for $p_\mathrm{T,\,Lead}$ and $p_\mathrm{T,\,Jet}$. The ridge yield for events with $p_\mathrm{T,\,jet} > 10$ GeV/$c$ is increased as $p_\mathrm{T,\,Lead}$ and $p_\mathrm{T,\,Jet}$ increase. The results are compared with PYTHIA 8 String Shoving model. The model describes the enhancement the ridge yield, however, underestimating the absolute ridge yield than data. This observation is important for the study of the origins of the ridge in small systems.

