% !TEX root = paper.tex

\section{Conclusions}
\label{sec:summary}
Long-range azimuthal correlations were studied in high-multiplicity pp collisions at $\sqrt{s} = 13$~TeV. These results were obtained in the relative pseudorapidity range $1.6 < |\Delta\eta| < 1.8$ and transverse momentum interval $1 < \pt < 5$~GeV/$c$ for the trigger and associated particles. 
The results were compatible to the measurements by the CMS collaboration~\cite{Khachatryan:2015lva}.
%In addition, the $\pt$ and the $\it{p}_{\rm{T,Lead}}$ or $\it{p}_{\rm{T,Jet}}$ selection dependence of the long-range azimuthal correlations are measured.
$\pythiashoving$ describes the yield qualitatively but the predicted yield decreases more rapidly than the data as $\pttrigassoc$ increases. On the other hand, $\epos$ gives better description for $\pt$ dependence while overestimating the ridge yield in the low $\pt<$2 GeV/$c$. $\pythiam$ does not describe the ridge in the long-range.
%Furthermore the ridge yields are studied in events, where jets with a harder fragmentation are preferred by requiring the minimum value of the transverse momentum of leading particles in each event $\ptlead$ or the one of reconstructed jets $\ptjet$. The ridge structure still persist with both selections. The ridge yields increase as $\ptlead$ and $\ptjet$ increase. The latter, however, is more significant than the former selection. The results are compared with $\pythiashoving$ calculations, showing the enhancement while underestimating the absolute ridge yield.
Furthermore the ridge yields were studied with the event-scale selections through the minimum transverse momentum of jet or leading particle requirement. 
The ridge structure still persists with both selections. The ridge yields increase as $\ptlead$ and $\ptjet$ increase. The results are compared with $\pythiashoving$, $\epos$ and $\pythiam$ calculations. $\pythiashoving$ and $\epos$ estimate qualitative trends for event-scale selection. However, the former underestimates the ridge yield and the latter overestimates the ridge yield.
Responses to event-scale selection are further studied by measuring near-side jet-like correlations. $\epos$ provides better description for near-side jet-like correlations with the event-scale selection, while $\pythiashoving$ overestimates near-side jet-like correlations, which results in opposite trends between the ridge yield and near-side jet-like correlations.
The results might open new way of studying impact parameter dependence of the small systems with jet tagged events in the future and will help to explore possible physical origins of long-range correlations.
