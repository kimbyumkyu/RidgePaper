% !TEX root = paper.tex

\section{Conclusions}
\label{sec:summary}
Long-range azimuthal correlations are studied in high-multiplicity pp collisions at $\sqrt{s} = 13$~TeV . These results are obtained in a relative pseudorapidity range of $1.6 < |\eta| < 1.8$ and transverse momentum region of $1 < p_{\mathrm T} < 5$~GeV/$c$ for the trigger and associated particles and 0--0.1\% multiplicity event class. 
The results are compatible to the measurements by the CMS collaboration~\cite{Khachatryan:2015lva}.
%In addition, the $\pt$ and the $\it{p}_{\rm{T,Lead}}$ or $\it{p}_{\rm{T,Jet}}$ selection dependence of the long-range azimuthal correlations are measured.
$\pythiashoving$ describes the yield qualitatively but its yield gets decreased more rapidly than the data as $\pttrigassoc$ increases.
Furthermore the ridge yields are studied in events, where jets with harder fragmentations is to be present by requiring the $\it{p}_{\rm{T,Lead}}$ or $\it{p}_{\rm{T,Jet}}$. The ridge structure still persist with both selections. The ridge yield for the events for $\it{p}_{\rm{T,jet}} > 10$~GeV/$c$ and $\it{p}_{\rm{T,Lead}} > 9$~GeV/$c$ is enhanced as the $\it{p}_{\rm{T,Lead}}$ or $\it{p}_{\rm{T,Jet}}$ increases. The results are compared with $\pythiashoving$ calculations, showing the enhancement while underestimating the absolute ridge yield.
The results open new aspects of studying the small collision systems with jet tagged events in the future and will help to explore possible physical origins of long-range correlations.